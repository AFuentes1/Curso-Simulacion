\documentclass[12pt]{article}
\usepackage[spanish]{babel}
\usepackage[utf8]{inputenc}
\usepackage{graphicx}
\usepackage{float}
\usepackage{geometry}
\usepackage{amsmath}
\geometry{margin=2.5cm}

\title{Simulación de un Sistema de Colas en un Restaurante de Comida Rápida}
\author{Anthony David Fuentes Calvo \\ Carlos Antonio Murcia Mora}
\date{\today}

\begin{document}

\maketitle

%------------------------------------------------
\section{Configuración del Modelo}

Los costos por servidor utilizados en la simulación fueron:

\begin{itemize}
\item Freidora: \$200
\item Caja: \$500
\item Refrescos: \$750
\item Pollo: \$100
\end{itemize}

La distribución de llegadas se modeló como una distribución exponencial con media de 1 minuto. Las distribuciones de servicio fueron:

\begin{itemize}
\item Caja: Exponencial (media = 3 min)
\item Freidora: Exponencial (media = 2.5 min)
\item Refrescos: Exponencial (media = 1.5 min)
\item Pollo: Exponencial (media = 4 min)
\end{itemize}

Los tipos de pedido considerados fueron:

\begin{itemize}
\item Solo refresco (30\%)
\item Frito y refresco (35\%)
\item Pollo y refresco (25\%)
\item Combo completo (10\%)
\end{itemize}

La simulación se ejecutó con 2000 clientes, 200 de calentamiento, 30 réplicas y semilla 12345. El objetivo del sistema fue mantener la espera promedio en cola por debajo de 3 minutos.

%------------------------------------------------
\section{Validación Estadística de los Generadores}

Los generadores de números aleatorios utilizados en la simulación fueron validados mediante pruebas de bondad de ajuste. Las distribuciones continuas fueron evaluadas usando la prueba de Kolmogorov--Smirnov (KS), mientras que la distribución discreta fue evaluada mediante la prueba de ji-cuadrado ($\chi^2$). En todos los casos se utilizó un nivel de significancia $\alpha = 0.05$.

\subsection*{Prueba KS -- Exponencial}

\begin{itemize}
\item Estadístico: $D = 0.0079$
\item p-value: $0.1633$
\end{itemize}

\textbf{Conclusión:} Como $p$-value $> \alpha$, no se rechaza la hipótesis nula. Los tiempos generados son compatibles con una distribución exponencial con la media especificada, por lo que el generador es válido para modelar llegadas y tiempos de servicio.

\subsection*{Prueba KS -- Normal}

\begin{itemize}
\item Estadístico: $D = 0.00435$
\item p-value: $0.8412$
\end{itemize}

\textbf{Conclusión:} No se rechaza la hipótesis nula con amplia holgura. La muestra es altamente consistente con una distribución normal con los parámetros dados, sin evidencia de sesgo o truncamiento.

\subsection*{Prueba $\chi^2$ -- Binomial}

\begin{itemize}
\item Estadístico: $\chi^2 = 8.64$
\item p-value: $0.1244$
\end{itemize}

\textbf{Conclusión:} Dado que $p$-value $> \alpha$, no se rechaza la hipótesis nula. Las frecuencias observadas coinciden con una distribución binomial dentro de la variabilidad esperada, validando el modelo discreto de cantidad de órdenes.

\subsection*{Conclusión Global}

Se concluye que los generadores de números aleatorios utilizados en la simulación cumplen con las distribuciones teóricas especificadas. En todos los casos los p-values fueron mayores que $\alpha = 0.05$, por lo que no se rechazó la hipótesis nula de ajuste. Esto valida formalmente la base estocástica del modelo.

\subsection*{Nota Técnica}

No se aplicaron pruebas de bondad de ajuste sobre los tiempos en sistema ni tiempos en cola, ya que estas variables no son independientes ni siguen las distribuciones originales, sino que dependen de la dinámica del sistema de colas, congestión y paralelismo entre estaciones.

%------------------------------------------------
\section{Resultados del Caso Base}

El tiempo promedio en el sistema fue:
\[
W = 11.70 \text{ minutos}
\]

La espera total promedio en colas fue:
\[
W_q = 2.76 \text{ minutos}
\]

Por estación:
\begin{itemize}
\item Caja: 1.36 min
\item Freidora: 0.05 min
\item Refrescos: 0.96 min
\item Pollo: 0.72 min
\end{itemize}

Estos resultados cumplen con el objetivo del sistema, manteniendo tiempos de espera aceptables para los clientes.

%------------------------------------------------
\section{Análisis del Escenario 2(c): Presupuesto \$3000}

La configuración evaluada fue:
\[
\{\text{Caja}=2,\ \text{Freidora}=1,\ \text{Refrescos}=2,\ \text{Pollo}=3\}
\]

con un costo total de \$3000. Sin embargo, esta configuración produjo una espera promedio de aproximadamente:
\[
W_q \approx 1004.7 \text{ minutos}
\]

lo cual indica inestabilidad del sistema. En particular, la estación de freidora presentó una utilización mayor que 1, generando crecimiento ilimitado de la cola.

Una configuración estable alternativa dentro del mismo presupuesto es:
\[
\{\text{Caja}=2,\ \text{Freidora}=2,\ \text{Refrescos}=2,\ \text{Pollo}=2\}
\]

con costo total:
\[
2(500) + 2(200) + 2(750) + 2(100) = \$3000
\]

Esta distribución mantiene todas las estaciones con utilización menor que 1 y reduce significativamente el tiempo de espera promedio.

%------------------------------------------------
\section*{Resultados Gráficos}

\subsection*{Caso A}

\begin{figure}[H]
\centering
\includegraphics[width=0.45\textwidth]{outputs/figures/a_1_box.png}
\includegraphics[width=0.45\textwidth]{outputs/figures/a_1_cov.png}
\caption{Caso A1: Boxplot y Covarianza}
\end{figure}

\begin{figure}[H]
\centering
\includegraphics[width=0.45\textwidth]{outputs/figures/a_1_hist_abs.png}
\includegraphics[width=0.45\textwidth]{outputs/figures/a_1_hist_rel.png}
\caption{Caso A1: Histogramas absoluto y relativo}
\end{figure}

\begin{figure}[H]
\centering
\includegraphics[width=0.6\textwidth]{outputs/figures/a_1_sens_pollo.png}
\caption{Caso A1: Análisis de sensibilidad (pollo)}
\end{figure}

\begin{figure}[H]
\centering
\includegraphics[width=0.45\textwidth]{outputs/figures/a_2_box.png}
\includegraphics[width=0.45\textwidth]{outputs/figures/a_2_cov.png}
\caption{Caso A2: Boxplot y Covarianza}
\end{figure}

\begin{figure}[H]
\centering
\includegraphics[width=0.45\textwidth]{outputs/figures/a_2_hist_abs.png}
\includegraphics[width=0.45\textwidth]{outputs/figures/a_2_hist_rel.png}
\caption{Caso A2: Histogramas absoluto y relativo}
\end{figure}

\begin{figure}[H]
\centering
\includegraphics[width=0.6\textwidth]{outputs/figures/a_2_sens_pollo.png}
\caption{Caso A2: Análisis de sensibilidad (pollo)}
\end{figure}

%------------------------------------------------
\subsection*{Caso B}

\begin{figure}[H]
\centering
\includegraphics[width=0.45\textwidth]{outputs/figures/b_1_box.png}
\includegraphics[width=0.45\textwidth]{outputs/figures/b_1_cov.png}
\caption{Caso B1: Boxplot y Covarianza}
\end{figure}

\begin{figure}[H]
\centering
\includegraphics[width=0.45\textwidth]{outputs/figures/b_1_hist_abs.png}
\includegraphics[width=0.45\textwidth]{outputs/figures/b_1_hist_rel.png}
\caption{Caso B1: Histogramas absoluto y relativo}
\end{figure}

\begin{figure}[H]
\centering
\includegraphics[width=0.45\textwidth]{outputs/figures/b_2_box.png}
\includegraphics[width=0.45\textwidth]{outputs/figures/b_2_cov.png}
\caption{Caso B2: Boxplot y Covarianza}
\end{figure}

\begin{figure}[H]
\centering
\includegraphics[width=0.45\textwidth]{outputs/figures/b_2_hist_abs.png}
\includegraphics[width=0.45\textwidth]{outputs/figures/b_2_hist_rel.png}
\caption{Caso B2: Histogramas absoluto y relativo}
\end{figure}

\begin{figure}[H]
\centering
\includegraphics[width=0.45\textwidth]{outputs/figures/b_3_box.png}
\includegraphics[width=0.45\textwidth]{outputs/figures/b_3_cov.png}
\caption{Caso B3: Boxplot y Covarianza}
\end{figure}

\begin{figure}[H]
\centering
\includegraphics[width=0.45\textwidth]{outputs/figures/b_3_hist_abs.png}
\includegraphics[width=0.45\textwidth]{outputs/figures/b_3_hist_rel.png}
\caption{Caso B3: Histogramas absoluto y relativo}
\end{figure}

%------------------------------------------------
\subsection*{Caso C}

\begin{figure}[H]
\centering
\includegraphics[width=0.45\textwidth]{outputs/figures/c_1_box.png}
\includegraphics[width=0.45\textwidth]{outputs/figures/c_1_cov.png}
\caption{Caso C1: Boxplot y Covarianza}
\end{figure}

\begin{figure}[H]
\centering
\includegraphics[width=0.45\textwidth]{outputs/figures/c_1_hist_abs.png}
\includegraphics[width=0.45\textwidth]{outputs/figures/c_1_hist_rel.png}
\caption{Caso C1: Histogramas absoluto y relativo}
\end{figure}

\begin{figure}[H]
\centering
\includegraphics[width=0.45\textwidth]{outputs/figures/c_2_box.png}
\includegraphics[width=0.45\textwidth]{outputs/figures/c_2_cov.png}
\caption{Caso C2: Boxplot y Covarianza}
\end{figure}

\begin{figure}[H]
\centering
\includegraphics[width=0.45\textwidth]{outputs/figures/c_2_hist_abs.png}
\includegraphics[width=0.45\textwidth]{outputs/figures/c_2_hist_rel.png}
\caption{Caso C2: Histogramas absoluto y relativo}
\end{figure}

\begin{figure}[H]
\centering
\includegraphics[width=0.45\textwidth]{outputs/figures/c_3_box.png}
\includegraphics[width=0.45\textwidth]{outputs/figures/c_3_cov.png}
\caption{Caso C3: Boxplot y Covarianza}
\end{figure}

\begin{figure}[H]
\centering
\includegraphics[width=0.45\textwidth]{outputs/figures/c_3_hist_abs.png}
\includegraphics[width=0.45\textwidth]{outputs/figures/c_3_hist_rel.png}
\caption{Caso C3: Histogramas absoluto y relativo}
\end{figure}

%------------------------------------------------
\subsection*{Caso D}

\begin{figure}[H]
\centering
\includegraphics[width=0.45\textwidth]{outputs/figures/d_1_box.png}
\includegraphics[width=0.45\textwidth]{outputs/figures/d_1_cov.png}
\caption{Caso D1: Boxplot y Covarianza}
\end{figure}

\begin{figure}[H]
\centering
\includegraphics[width=0.45\textwidth]{outputs/figures/d_1_hist_abs.png}
\includegraphics[width=0.45\textwidth]{outputs/figures/d_1_hist_rel.png}
\caption{Caso D1: Histogramas absoluto y relativo}
\end{figure}

\begin{figure}[H]
\centering
\includegraphics[width=0.6\textwidth]{outputs/figures/d_1_sens_pollo.png}
\caption{Caso D1: Análisis de sensibilidad (pollo)}
\end{figure}

%------------------------------------------------
\subsection*{Caso E}

\begin{figure}[H]
\centering
\includegraphics[width=0.45\textwidth]{outputs/figures/e_1_box.png}
\includegraphics[width=0.45\textwidth]{outputs/figures/e_1_cov.png}
\caption{Caso E1: Boxplot y Covarianza}
\end{figure}

\begin{figure}[H]
\centering
\includegraphics[width=0.45\textwidth]{outputs/figures/e_1_hist_abs.png}
\includegraphics[width=0.45\textwidth]{outputs/figures/e_1_hist_rel.png}
\caption{Caso E1: Histogramas absoluto y relativo}
\end{figure}

\begin{figure}[H]
\centering
\includegraphics[width=0.6\textwidth]{outputs/figures/e_1_sens_pollo.png}
\caption{Caso E1: Análisis de sensibilidad (pollo)}
\end{figure}

\begin{figure}[H]
\centering
\includegraphics[width=0.45\textwidth]{outputs/figures/e_2_box.png}
\includegraphics[width=0.45\textwidth]{outputs/figures/e_2_cov.png}
\caption{Caso E2: Boxplot y Covarianza}
\end{figure}

\begin{figure}[H]
\centering
\includegraphics[width=0.45\textwidth]{outputs/figures/e_2_hist_abs.png}
\includegraphics[width=0.45\textwidth]{outputs/figures/e_2_hist_rel.png}
\caption{Caso E2: Histogramas absoluto y relativo}
\end{figure}

\begin{figure}[H]
\centering
\includegraphics[width=0.6\textwidth]{outputs/figures/e_2_sens_pollo.png}
\caption{Caso E2: Análisis de sensibilidad (pollo)}
\end{figure}

\begin{figure}[H]
\centering
\includegraphics[width=0.45\textwidth]{outputs/figures/e_3_box.png}
\includegraphics[width=0.45\textwidth]{outputs/figures/e_3_cov.png}
\caption{Caso E3: Boxplot y Covarianza}
\end{figure}

\begin{figure}[H]
\centering
\includegraphics[width=0.45\textwidth]{outputs/figures/e_3_hist_abs.png}
\includegraphics[width=0.45\textwidth]{outputs/figures/e_3_hist_rel.png}
\caption{Caso E3: Histogramas absoluto y relativo}
\end{figure}

\begin{figure}[H]
\centering
\includegraphics[width=0.6\textwidth]{outputs/figures/e_3_sens_pollo.png}
\caption{Caso E3: Análisis de sensibilidad (pollo)}
\end{figure}

\end{document}
